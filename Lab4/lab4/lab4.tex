\documentclass{article}
\usepackage{amsmath, amsfonts, amsthm, amssymb}  

\usepackage{secdot}
\usepackage{textcomp}
\usepackage{epsfig}
\usepackage{cprotect}
\usepackage[T1]{fontenc}
\usepackage{epstopdf}
\usepackage{hyperref}
\usepackage{rotating}
\usepackage{graphicx}
\usepackage{caption}
\usepackage{subcaption}
\usepackage{multirow}
\usepackage{setspace}
\usepackage{array}
\usepackage{fancyhdr}
\usepackage{lastpage}
\usepackage[T1]{fontenc}

\usepackage{geometry}
\geometry{letterpaper, left=1in, right=1in, top=1in, bottom=1in}

\pagestyle{fancy}
\fancyhf{}
\rhead{\thepage/\pageref{LastPage}}
\lhead{OSU ECEN 2233 - Logic Design - Fall 2023}
\rfoot{\LaTeX}


% ----- Identifying Information -----------------------------------------------
\newcommand{\myassignment}{Lab 4: Using IP and Ethernet}
\newcommand{\myduedate}{Assigned: Monday 10/23; Due \textbf{Monday 11/6} (midnight)}
\newcommand{\myinstructor}{Instructor: James E. Stine, Jr.}
% -----------------------------------------------------------------------------

\begin{document}
\begin{center}
  {\huge \myassignment} \\
  {\large \myduedate} \\
  \begin{flushright}
  \myinstructor \\
  \end{flushright}
\end{center}

\section{Introduction}

Ethernet was developed at Xerox PARC between 1970s as a
method to allow Alto computers to communicate with each other. It
was inspired by the ALOHAnet, which Robert Metcalfe had studied as part of
his PhD dissertation and was originally called the Alto Aloha
Network.  Just like many engineering advancements, Robert Metcalfe and
his colleagues were in the right place at the right time in addition
to trying an idea that is practical and easy to implement.  I guess
the adage is the rest is history.
A really great YouTube video documenting the ideas of Ethernet can be
found here~\url{https://youtu.be/g5MezxMcRmk?si=UomwAQHmMgY6cvh4}.  

Personally, I think the idea is quite efficient and straight forward,
but it is so vital to today's society that I firmly believe its
something you should know about.  This laboratory will be about using
Ethernet from some pre-built Hardware Descriptive Language (HDL) or
sometimes called Intellectual Property (IP).  In many digital areas,
IP is essential to many products both commercially and in research
that you have to just add it to your design and try it.  This
laboratory will be about that very idea -- that is, to use specific IP
and understand how to use it.

\subsection{Ethernet Packets}

Ethernet is the most widely used LAN technology and is defined under
the IEEE standards 802.3. The reason behind its wide usability is that
Ethernet is easy to understand, implement, and maintain, and, most
importantly, allows a
low-cost network implementation. Also, Ethernet offers flexibility in
terms of the topologies that are allowed. Ethernet generally uses a
bus topology where each item connects to the Ether.
Ethernet operates in two layers of the OSI model, the
physical layer and the data link layer. For Ethernet, the protocol
data unit is a frame since we mainly deal with DLLs. In order to
handle collisions, the Access control mechanism used in Ethernet is
CSMA/CD.

Although Ethernet has been largely replaced by wireless networks,
wired networking still uses Ethernet more frequently. Wi-Fi eliminates
the need for cables by enabling users to connect their smartphones or
laptops to a network wirelessly. The 802.11ac Wi-Fi standard offers
faster maximum data transfer rates when compared to Gigabit
Ethernet. However, wired connections are more secure and less
susceptible to interference than wireless networks. This is the main
justification for why so many companies and organizations continue to
use Ethernet.

The basic frame format which is required for all MAC implementation is
defined in IEEE 802.3 standard. Though several optional formats are
being used to extend the protocol’s basic capability. Ethernet frame
starts with Preamble and SFD, both work at the physical
layer. Ethernet header contains both the Source and Destination MAC
address, after which the payload of the frame is present. The last
field is CRC which is used to detect the error. Now, let’s study each
field of basic frame format.

The size of an Ethernet frame by the IEEE 802.3 varies from $64$ bytes
to $1,518$ bytes including the data length (i.e., $46$ to $1,500$
bytes).  T

% https://www.geeksforgeeks.org/ethernet-frame-format/
\begin{itemize}
\item PRE (preamble) – Ethernet frame starts with a $7$-Bytes preamble. This
  is a pattern of alternative $0$’s and $1$’s which indicates starting of
  the frame and allow sender and receiver to establish bit
  synchronization. Initially, the PRE (Preamble) is introduced to allow
  for the loss of a few bits due to signal delays. But today’s
  high-speed Ethernet doesn’t need the Preamble to protect the frame
  bits. PRE (Preamble) indicates the receiver that frame is coming and
  allow the receiver to lock onto the data stream before the actual
  frame begins.  
\item Start of frame delimiter (SFD) – This is a $1$-Byte field that is
  always set to \verb!1010_1011!. SFD indicates that upcoming bits are
  starting the frame, which is the destination address. Sometimes SFD
  is considered part of PRE, this is the reason Preamble is described
  as 8 Bytes in many places. The SFD warns station or stations that
  this is the last chance for synchronization.  
\item Destination Address – This is a $6$-Byte field that contains the
  MAC address of the machine for which data is destined.  
\item Source Address – This is a $6$-Byte field that contains the MAC
  address of the source machine. As Source Address is always an
  individual address (Unicast), the least significant bit of the first
  byte is always $0$.  
\item Length – Length is a $2$-Byte field, which indicates the length of
  the entire Ethernet frame. This 16-bit field can hold a length value
  between $0$ to $65,534$, but length cannot be larger than $1,500$ Bytes
  because of some own limitations of Ethernet.  
\item Data – This is the place where actual data is inserted, also
  known as Payload. Both IP header and data will be inserted here if
  Internet Protocol is used over Ethernet. The maximum data present
  may be as long as $1,500$ Bytes. In case data length is less than
  minimum length (i.e., $46$ bytes), then padding $0$’s is added to meet the
  minimum possible length.
\item Cyclic Redundancy Check (CRC) – CRC is $4$ Byte field. This field
  contains a $32$-bits hash code of data, which is generated over the
  Destination Address, Source Address, Length, and Data field. If the
  checksum computed by destination is not the same as sent checksum
  value, data received is corrupted.  
\item VLAN Tagging – The Ethernet frame can also include a VLAN
  (Virtual Local Area Network) tag, which is a $4$-byte field inserted
  after the source address and before the EtherType field. This tag
  allows network administrators to logically separate a physical
  network into multiple virtual networks, each with its own VLAN ID.  
\item Jumbo Frames – In addition to the standard Ethernet frame size
  of $1,518$ bytes, some network devices support Jumbo Frames, which are
  frames with a payload larger than $1,500$ bytes. Jumbo Frames can
  increase network throughput by reducing the overhead associated with
  transmitting a large number of small frames.  
\item Ether Type Field – The EtherType field in the Ethernet frame
  header identifies the protocol carried in the payload of the
  frame. For example, a value of \verb!0x0800! indicates that the payload is
  an IP packet, while a value of \verb!0x0806! indicates that the payload is
  an ARP (Address Resolution Protocol) packet.  
\item Multicast and Broadcast Frames –  In addition to Unicast frames
  (which are sent to a specific destination MAC address), Ethernet
  also supports Multicast and Broadcast frames. Multicast frames are
  sent to a specific group of devices that have joined a multicast
  group, while Broadcast frames are sent to all devices on the
  network.  
\item Collision Detection – In half-duplex Ethernet networks,
  collisions can occur when two devices attempt to transmit data at
  the same time. To detect collisions, Ethernet uses a Carrier Sense
  Multiple Access with Collision Detection (CSMA/CD) protocol, which
  listens for activity on the network before transmitting data and
  backs off if a collision is detected.
\end{itemize}
  







\bibliographystyle{IEEEbib}
\bibliography{lab4}

\end{document}
